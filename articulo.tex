\documentclass[twoside,twocolumn]{article}

\usepackage{blindtext} 
\usepackage{graphicx}
\usepackage[sc]{mathpazo} 
\usepackage[T1]{fontenc} 
\linespread{1.05} 
\usepackage{microtype} 

\usepackage[english]{babel} 

\usepackage[hmarginratio=1:1,top=32mm,columnsep=20pt]{geometry} 
\usepackage[hang, small,labelfont=bf,up,textfont=it,up]{caption} 
\usepackage{booktabs} 

\usepackage{lettrine} 

\usepackage{enumitem} 
\setlist[itemize]{noitemsep} 

\usepackage{abstract} 
\renewcommand{\abstractnamefont}{\normalfont\bfseries} 
\renewcommand{\abstracttextfont}{\normalfont\small\itshape} 

\usepackage{titlesec} 
\renewcommand\thesection{\Roman{section}} % 
\renewcommand\thesubsection{\roman{subsection}} 
\titleformat{\section}[block]{\large\scshape\centering}{\thesection.}{1em}{} 
\titleformat{\subsection}[block]{\large}{\thesubsection.}{1em}{} 

\usepackage{fancyhdr} 
\pagestyle{fancy} 
\fancyhead{} 
\fancyfoot{} 
\fancyhead[C]{Titulo $\bullet$ Junio 2019 $\bullet$ } 
\fancyfoot[RO,LE]{\thepage} 

\usepackage{titling} 

\usepackage{hyperref} 

%----------------------------------------------------------------------------------------
%	TILULOS
%----------------------------------------------------------------------------------------

\setlength{\droptitle}{-4\baselineskip} 

\pretitle{\begin{center}\Huge\bfseries} 
\posttitle{\end{center}} 
\title{Titulo del articulo} 
\author{Andre Reinoso, }
\date{\today} 
\renewcommand{\maketitlehookd}{
\begin{abstract}
\noindent 
Abstract en ingles
\end{abstract}
\begin{abstract}
\noindent 
Abstract en español
\end{abstract}
}

%----------------------------------------------------------------------------------------

\begin{document}

% Print the title
\maketitle

%----------------------------------------------------------------------------------------
%	INTRODUCCION
%----------------------------------------------------------------------------------------

\section{Introduccion}
\lettrine[nindent=0em,lines=3]{A}ctualmente en el Perú las pequeñas y medianas empresas producen al mercado peruano ingresos y empleado, la gran cantidad de informacion que manejan es debido al alto numero de operaciones que realizan a diario.\\ \\
El aumento de competencia hace que el area administrativa tome decisiones a base de su experiencia, publicidad, tecnologia, recursos humanos y hasta de su propia intuicion.\\ \\
Sin embargo hay empresas que no toman de manera estructurada las decisiones, o no implementan herramientas de Business Intelligence que permitirian al gerente escenarios, pronosticos, reportes, etc. una mejor toma de deciciones respecto a su rubro. \\ \\
Aquella empresa que si lleva a cabo procesos de extraccion de datos, transformacion de datos, uso de herramientas y metodos, tendra una ventaja notoria a comparacion de las demas que no lo aplican.


%----------------------------------------------------------------------------------------
%	Objetivos
%----------------------------------------------------------------------------------------

\section{Marco teorico}

\begin{itemize}
\item Aqui
\item Aqui

\end{itemize}



%----------------------------------------------------------------------------------------
%	DESARROLLO
%----------------------------------------------------------------------------------------

\section{Desarrollo}

\subsection{¿Que es la inteligencia de negocios?}

La informacion ha propiciado la necesidad de tener mejores, mas rapidos y mas eficientes metodos para extraer y transformar los datos de una organizacion en informacion.
\\ \\
En 1958 \textbf{Hans Peter Luhn}, investigador de la IBM, acuña el termino en el articulo "A Business Intelligence System":
\\ \\
\textsl{''La habilidad de aprender las relaciones de hechos presentados de forma que guien las acciones hacia una meta deseada''}
\\ \\
En 1989 \textbf{Howard Dresden}, analista de Gartner, propone una definicion formal del concepto.
\\ \\
\textsl{"Conceptos y métodos para mejorar la toma de decisiones comerciales mediante el uso de un sistema de soporte basado en hechos''}
\\ \\
En pocas palabras: \\ \\
\textbf{Es el conjunto de metedologias, aplicaciones, capacidades administrativas de informacion que permite tomar mejores decisiones a los usuarios de una organizacion.}


\subsection{¿Cuando aplicar la inteligencia de negocios?}

Contenido

\subsection{Titulo Nr3}

Contenido
 
\subsection{Beneficions de la inteligencia de negocios}

PONGAN DESCRIPCION Y EJEMPLOS min 2

\subsection{Titulo Nr5}

Contenido

\subsection{Titulo Nr6}
Contenido

\subsection{Titulo Nr7l}
Contenido

\begin{itemize}	

	\item Lista 1
	\item Lista 2

\end{itemize} 



\subsection{Titulo Nr8}
Contenido 


\subsection{Titulo Nr9}
Contenido 



\subsection{Titulo Nr10}
Contenido 

\begin{center}
	\includegraphics[width=7cm]{./Imagenes/img} 
\end{center}

%----------------------------------------------------------------------------------------
%	CONCLUSIONES
%----------------------------------------------------------------------------------------

\section{Conclusiones}

Aqui van las conclusiones


%----------------------------------------------------------------------------------------
%	BIBLIOGRAFIA
%----------------------------------------------------------------------------------------

\begin{thebibliography}{99} 

\bibitem[Silvia Chavez y Carmen Contreras, 2018]{}
\newblock Implementación de Business Intelligence, para el proceso de toma de decisiones del área de ventas.

\bibitem[Jordi Conesa y Josep Curto, 2010]{}
\newblock Introduccion al Business Intelligence

\bibitem[Nombre autor, fecha]{}
\newblock descripcion de la referencia
 

 
\end{thebibliography}

%----------------------------------------------------------------------------------------

\end{document}
